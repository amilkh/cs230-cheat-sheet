\section{End-to-End Deep Learning}

Pros of end-to-end deep learning:

\begin{itemize}[wide, labelwidth=!, labelindent=0pt]
\itemsep0em 
    \item Let the data speak. By having a pure machine learning approach, your NN learning input from X to Y may be more able to capture whatever statistics are in the data, rather than being forced to reflect human preconceptions.
    \item Less hand-designing of components needed. \vspace*{-\baselineskip}
\end{itemize}


Cons of end-to-end deep learning:

\begin{itemize}[wide, labelwidth=!, labelindent=0pt]
\itemsep0em 
    \item May need a large amount of data.
    \item Excludes potentially useful hand-design components (it helps more on the smaller dataset). \vspace*{-\baselineskip}
\end{itemize}

Applying end-to-end deep learning:

\begin{itemize}[wide, labelwidth=!, labelindent=0pt]
\itemsep0em 
    \item Key question: Do you have sufficient data to learn a function of the complexity needed to map x to y?
    \item Use ML/DL to learn some individual components.
    \item When applying supervised learning you should carefully choose what types of X to Y mappings you want to learn depending on what task you can get data for. \vspace*{-\baselineskip}
\end{itemize}