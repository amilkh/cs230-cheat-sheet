\section{Comparing to Human Level Performance}

We compare to human-level performance because of two main reasons:

\begin{itemize}[wide, labelwidth=!, labelindent=0pt]
\itemsep0em 
    \item Because of advances in deep learning, machine learning algorithms are suddenly working much better and so it has become much more feasible in a lot of application areas for machine learning algorithms to actually become competitive with human-level performance.
    \item It turns out that the workflow of designing and building a machine learning system is much more efficient when you're trying to do something that humans can also do. After an algorithm reaches the human level performance the progress and accuracy slow down. \vspace*{-\baselineskip}
\end{itemize}

\textbf{Bayes Optimal Error: } Bayes Optimal Error is the theoretical best possible error we can achieve. Usually we use human error as a proxy for Bayes optimal error because humans are fairly good at a lot of tasks. As long as your model is worse than humans, you can: get labeled data from humans, gain insight from manual error analysis, and better analyze bias and variance.

\textbf{Avoidable Bias: } Defined as the error between train error and human error. Methods for improving avoidable bias are the same as for improving bias.

